\documentclass[12pt,a4paper]{article}
\usepackage[utf8]{inputenc}
\usepackage[brazilian]{babel}
\usepackage{indentfirst}
\usepackage{lipsum}

\setlength{\parskip}{0.5em}
\renewcommand{\baselinestretch}{1.1}

\title{Sistema Web para Instalação de ERBs}
\author{Mateus Nakajo de Mendonça  \\
	Eric Rodrigues Pires  \\
    Orientador: Bruno de Carvalho Albertini
	}

\date{28 de janeiro de 2018}

\begin{document}
\maketitle

\begin{abstract}

%@TODO: Resumo
\lipsum[1]

\end{abstract}

\section{Introdução}

%@TODO: Explicar motivações (setor de instalação de ERBs no Brasil, em especial
%para aluguel, e necessidade de automação de seus estudos)
\lipsum[2-3]

\section{Objetivo}

O objetivo deste projeto de formatura é criar um sistema que permita calcular
posições para a instalação de antenas de telefonia de forma a maximizar o
alcance delas. Com esse fim, levaremos em conta dados geográficos para
realizarmos os cálculos.

Também é de grande importância que tal sistema tenha uma interface prática
para os usuários. Portanto, uma interface web que apresente os dados
requisitados é essencial para o projeto.

\subsection{Sistema de Informação Geográfica}

%@TODO: Definições do SIG (ex.: posição de ERBs atuais, relevo,
%consumidores atingidos) e programação linear
\lipsum[4-5]

%Levaremos em conta as antenas pré-existentes e o relevo da região
%em questão para fazermos o cálculo. Usaremos um software de SIG (Sistema de
%Informação Geográfica) para representar, analisar, e visualizar as informações
%geográficas.

\subsection{Interface Web}

%@TODO: Definições da interface web (ex.: responsividade, simplicidade,
%OpenStreetMap)
\lipsum[6-7]

%Para interação com o usuário, criaremos uma aplicação Web que o
%permita selecionar a região na qual ele pretende instalar alguma antena.

\end{document}

