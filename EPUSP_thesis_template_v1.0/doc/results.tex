\chapter{Main results} \label{chap:res}

In the remaining paragraphs, we first prove in Proposition \ref{p:optimality} that the optimal solution of problem $PU(\nu,\xi)$ will also be an optimal solution of problem $PC(\nu,\beta,\alpha)$ whenever $\xi=m \beta$, with $m$ positive and $\sum_{t=1}^{T} \nu(t)Var\left[y^{u^*}(t)\right] = \alpha$.
Then, in Theorem \ref{t:PC3}, we formalize how to obtain the explicit optimal control policy of problem $PC(\nu,\beta,\alpha)$. %the conditions that yield

% ****************************************	Proposition optimality
\begin{prop} \label{p:optimality}
	Suppose that in problem $PU(\nu,\xi)$ we have that $\xi=m\beta$ holds for some $m$ $\in$ $\mathbb{R^+}$.
	If $u^{*}$ is an optimal solution of problem $PU(\nu,\xi)$ such that  
		$\sum_{t=1}^{T} \nu(t)Var\left[y^{u^*}(t)\right] = \alpha$,
	then $u^{*}$ is an optimal solution of problem $PC(\nu,\beta,\alpha)$.
\end{prop}

\begin{proof}
	Suppose by contradiction that $u^{*}$ is \emph{not} an optimal solution of $PC(\nu,\beta,\alpha)$. %the consequent of Proposition \ref{p:optimality} is not true, i.e, 
	On the other hand, suppose $u^{*}$ is an optimal solution of problem $PU(\nu,\xi)$, i.e.,
	\begin{flalign}	\label{a1}
		\mathcal{C}(u^{\ast })=\min_{u\in \mathbb{U}} \mathcal{C}(u),
	\end{flalign}			
	%
	such that 
	\begin{flalign}	\label{a2}
		\sum_{t=1}^{T} \nu(t)Var\left[y^{u^*}(t)\right] = \alpha.
	\end{flalign}	

	Lets also assume that there exists an optimal solution of $PC(\nu,\beta,\alpha)$, named  $u$, and $u\in \mathbb{U}$.
	Hence, due to the optimality of $u$ and from Equation (\ref{a2}), we can state that
%
	\begin{flalign}	\label{h1}	
		&\sum_{t=1}^{T} \nu(t) Var\left[y^{u}(t)\right] \leqslant 
		\sum_{t=1}^{T} \nu(t)Var\left[y^{u^*}(t)\right] \Leftrightarrow \nonumber \\
		&\sum_{t=1}^{T} \nu(t) Var\left[y^{u}(t)\right] \leqslant  \alpha
	\end{flalign}	
and 
	\begin{flalign}	\label{h2}
		\sum_{t=1}^{T} \beta(t)E\left[y^{u}(t)\right] > \sum_{t=1}^{T} 
		\beta(t)E\left[y^{u^{*}} (t)\right]. 
	\end{flalign}
	
	Therefore, multiplying both sides of the Inequality (\ref{h2})  by $m>0$, and recalling that $\xi=m\beta$, we obtain 
	\begin{flalign}	\label{h3}
		&\sum_{t=1}^{T} m\beta(t)E\left[y^{u}(t)\right] > \sum_{t=1}^{T} m\beta(t)E\left[y^{u^{*}} (t)\right] \Leftrightarrow \nonumber \\
		&\sum_{t=1}^{T} \xi(t)E\left[y^{u}(t)\right] > \sum_{t=1}^{T} \xi(t)E\left[y^{u^{*}} (t)\right].
	\end{flalign}

	Finally, from (\ref{costU}) we have that the cost of the control law $u$ for problem $PU(\nu,\xi)$ is given by
	%
	\begin{flalign} \label{cost1}
		\mathcal{C}(u) = \sum_{t=1}^{T} \biggl[\nu(t)Var\big[ y^{u}(t) \big] 
			-\xi(t)E\big[ y^{u}(t) \big] \biggr].
	\end{flalign}
	%
	Substituting (\ref{h1}) and (\ref{h3}) into (\ref{cost1}) we have that
	\begin{flalign} \label{cost2}
		&\mathcal{C}(u) \leqslant \alpha - \sum_{t=1}^{T} \xi(t)E\big[ y^{u}(t) \big] 
		\Leftrightarrow \nonumber \\
		& \mathcal{C}(u) < \alpha - \sum_{t=1}^{T} \xi(t)E\big[ y^{u^{*}}(t) \big],
	\end{flalign}
	%
	and using (\ref{a2}) into (\ref{cost2}) we obtain that
	%, that $\sum_{t=1}^{T} \nu(t)Var\left[y^{u^*}(t)\right] = \alpha$
	\begin{flalign} \label{cost3}
		&\mathcal{C}(u) < \sum_{t=1}^{T} \biggl[ \nu(t)Var\big[ y^{u^{*}}(t) \big] 
			-\xi(t)E\big[ y^{u^{*}}(t) \big] \biggr]
		= \mathcal{C}(u^{*}).
	\end{flalign}
	%
	Thus, $\mathcal{C}(u)<\mathcal{C}(u^{\ast })$ for some $u\in \mathbb{U}$, 
	in contradiction to Hypothesis (\ref{a1}).
	Therefore, $u^{*}$ is an optimal solution of problem $PC(\nu,\beta,\alpha)$.
\end{proof}

% ****************************************	theorem PC3

We will make the following set of definitions in order to present Theorem \ref{t:PC3}:
%
\begin{flalign}  \label{r1}
	&\mathbb{W}_1 \in \mathbb{B}(\mathbb{R}^T): \;
		\mathbb{W}_1 = [ (I-2\tilde{\mathbb{C}})^{-1} ]' 
		[ \frac{1}{2}I - \tilde{\mathbb{C}}' ]\tilde{\mathbb{B}}
		(I-2\tilde{\mathbb{C}})^{-1} ,  \nonumber \\
	&\mathbb{W}_2 \in \mathbb{B}(\mathbb{R},\mathbb{R}^T): \;
		\mathbb{W}_2 = -2 a'\tilde{\mathbb{C}} 
		(I-2\tilde{\mathbb{C}})^{-1},  \nonumber \\
	&r_{1} \in \mathbb{R}: \; 
		 r_{1} = \beta'\mathbb{W}_1 \beta,  \nonumber \\
	&r_{2} \in \mathbb{R}: \; %& \nonumber \\ & \quad
		 r_{2} = -( 2\eta'\mathbb{W}_1' + 2\eta'\mathbb{W}_1 + 
		 \mathbb{W}_2 ) \beta,  \nonumber \\ 
	&r_{3} \in \mathbb{R}: \;
		 r_{3} = 4\eta'\mathbb{W}_1\eta + 2\mathbb{W}_2\eta  
			+ c-\alpha -\eta'a, \nonumber \\
	& f(\beta,\alpha) \in \mathbb{R}: \;  \boldsymbol{ f(\beta,\alpha) = \frac{r_2 + \sqrt[2]{r_{2}^2-4r_1r_3}}{2r_1} }.
\end{flalign}


\begin{theorem} \label{t:PC3} 
	Suppose Assumption \ref{aH} holds, $ \tilde{\mathbb{B}} - 
	2 \tilde{\mathbb{B}} \Gamma \tilde{\mathbb{B}} > 0$, and 
	$ f(\beta,\alpha) > 0$.
	Let
	\begin{flalign} \label{tPC3:e1}
		\xi = f(\beta,\alpha) \beta.
	\end{flalign}
	Then an optimal control strategy $u^{\lambda}$ for problem 	
	$PC(\nu,\beta,\alpha)$ is given as in (\ref{u_k}) with $\lambda$ as in 
	(\ref{PU:lambda}) and $\xi$ as in (\ref{tPC3:e1}).
\end{theorem}

\begin{proof}
	Given that $u^{\lambda}$ is an optimal solution of $PU(\nu,\xi)$,
	we have from Equation (\ref{eq:SvarVector}) that 
	$\sum_{t=1}^{T} \nu(t)Var\bigl[ y^{u^{\lambda}}(t)  \bigr]  = \alpha$ 
	is equivalent to 
	%
	\begin{flalign} \label{r2}	
		\alpha = \lambda'\left( \frac{1}{2} I - 
		\tilde{\mathbb{C}}' \right)\tilde{\mathbb{B}}\lambda  
		- 2 \eta'\tilde{\mathbb{B}} \lambda + c -\eta'a.
	\end{flalign}
	
	It follows that, applying $\lambda$ from (\ref{PU:lambda}) into the previous 
	Equation (\ref{r2})  %holds for some $m$ $\in$ $\mathbb{R^+}$, 
	we obtain the following quadratic equation with the unknown vector $\xi$:
	%
	\begin{flalign} \label{r3}
		&\lambda'\left( \frac{1}{2} I - \tilde{\mathbb{C}}'\right)\tilde{\mathbb{B}}\lambda  
			- 2 \eta'\tilde{\mathbb{B}} \lambda + c -\alpha -\eta'a = 0 \Leftrightarrow 
			\nonumber \\
		&(\xi'+2 a' \Gamma')[(I-2\Gamma\tilde{\mathbb{B}})^{-1}]' 
		 \left( \frac{1}{2} I -	 \tilde{\mathbb{C}}'\right)\tilde{\mathbb{B}}
		 (I-2\Gamma\tilde{\mathbb{B}})^{-1}(\xi+2\Gamma a)  \nonumber \\	
		& \qquad - 2 \eta'\tilde{\mathbb{B}} (I-2\Gamma\tilde{\mathbb{B}})^{-1}
		 (\xi+2\Gamma a)  + c -\alpha -\eta'a = 0.
	\end{flalign}
			
	After some algebraic manipulation and recalling the definitions of $\eta$ and $\tilde{\mathbb{C}}$ from (\ref{eta}) and (\ref{Ct}), respectively, Equation (\ref{r3}) becomes
	\begin{flalign}\label{r4}
		&\xi'[(I-2\tilde{\mathbb{C}})^{-1}]' 
		\left( \frac{1}{2} I - \tilde{\mathbb{C}}'\right)\tilde{\mathbb{B}}
		(I-2\tilde{\mathbb{C}})^{-1} \xi + 
		\xi'[(I-2\tilde{\mathbb{C}})^{-1}]' 
		\left( \frac{1}{2} I - \tilde{\mathbb{C}}'\right)\tilde{\mathbb{B}}
		(I-2\tilde{\mathbb{C}})^{-1} 2\eta  \nonumber \\
		& \qquad +2\eta'[(I-2\tilde{\mathbb{C}})^{-1}]' 
		\left( \frac{1}{2} I - \tilde{\mathbb{C}}'\right)\tilde{\mathbb{B}}
		(I-2\tilde{\mathbb{C}})^{-1} \xi + 
		2\eta'[(I-2\tilde{\mathbb{C}})^{-1}]' 
		\left( \frac{1}{2} I - \tilde{\mathbb{C}}'\right)\tilde{\mathbb{B}}
		(I-2\tilde{\mathbb{C}})^{-1} 2\eta \nonumber \\
		& \qquad - 2 a'\tilde{\mathbb{C}} (I-2\tilde{\mathbb{C}})^{-1} \xi
			- 4 a'\tilde{\mathbb{C}} (I-2\tilde{\mathbb{C}})^{-1} \eta
		 	 + c -\alpha -\eta'a = 0.
	\end{flalign}
	
	Substituting $\mathbb{W}_1$, $\mathbb{W}_2$, and $r_3$ from (\ref{r1}) into Equation (\ref{r4}) we obtain
	\begin{flalign}\label{r5}	
		& \xi' \mathbb{W}_1 \xi + 2\xi' \mathbb{W}_1 \eta + 2\eta' \mathbb{W}_1 \xi 
			+ 4 \eta' \mathbb{W}_1 \eta + \mathbb{W}_2 \xi + 2\mathbb{W}_2 \eta 
			+ c -\alpha -\eta'a = 0 \Leftrightarrow \nonumber \\
		& \xi' \mathbb{W}_1 \xi + 2\eta' \mathbb{W}_1' \xi + 2\eta' \mathbb{W}_1 \xi 
			 + \mathbb{W}_2 \xi + r_3= 0.
	\end{flalign}
%
	Assuming $\xi$ is a positive multiple of $\beta$, that is $\xi = m \beta$, and substituting the definitions of $r_1$ and $r_2$ from (\ref{r1}) into (\ref{r5}) we have that
	\begin{flalign}\label{r6}	
		& \beta' \mathbb{W}_1 \beta m^2 + 2 \eta' \mathbb{W}_1' \beta m
		+ 2\eta' \mathbb{W}_1 \beta m + \mathbb{W}_2 \beta m + r_3= 0 
			\Leftrightarrow \nonumber \\
		& \beta' \mathbb{W}_1 \beta m^2 + \left( 2\eta' \mathbb{W}_1' 
		+ 2\eta' \mathbb{W}_1  + \mathbb{W}_2 \right) \beta m + r_3= 0 
			\Leftrightarrow \nonumber \\
		& r_1 m^2 - r_2 m + r_3 = 0,
	\end{flalign}	
	whose positive solution is $m=f(\beta,\alpha)$ as defined in (\ref{r1}).
%
	Therefore, following the algorithm of Section \ref{remark:uk} by fixing $\xi$ in terms of the input parameters $\beta$ and $\alpha$ through $\xi = f(\beta,\alpha) \beta$, $f(\beta,\alpha) > 0$, and computing $\lambda$ with Equation (\ref{PU:lambda}),  we can obtain the optimal control law of problem $PU(\nu,\xi)$, $u^{\lambda}$, that yields $\sum_{t=1}^{T} \nu(t)Var\left[y^{u^{\lambda}}(t)\right] = \alpha$.
Hence, from Proposition \ref{p:optimality}, we have that $u^{\lambda}$ is also an optimal solution of $PC(\nu,\beta,\alpha)$, completing the proof.
\end{proof}
